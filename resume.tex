\documentclass[line,margin]{res}
	\usepackage{fancyhdr}
	\pagestyle{fancy}
	\renewcommand{\headrulewidth}{0pt}
	\usepackage[usenames,dvipsnames]{xcolor}
	\usepackage[colorlinks=true,linkcolor=blue,urlcolor=NavyBlue,citecolor=Fuchsia]{hyperref}

	\newsectionwidth{0pt}
	\setlength{\headsep}{24pt}
	\topmargin=-0.5in

	\lfoot {} \cfoot {}
	\rfoot {\scriptsize{Resume viewable online at \href{https://github.com/justaddbass/Resume/blob/master/resume.pdf}{github.com/justaddbass/Resume/blob/master/resume.pdf}}}


\begin{document}
	\name{Randall Blake}
	\address{(512)574-6609 |
	\href{mailto:blakerandall0@gmail.com}{blakerandall0@gmail.com}}

\begin{resume}

\vspace{8pt}
\section{Summary}
\vspace{22pt}
	\begin{itemize} \itemsep -2pt
%	\item Interested in pursuing a career in AI or computer graphics
	\item Experience with version control systems, continuous integration, and bug tracking
	\item Studying AI, computational neural networks, and computer graphics
	\end{itemize}

\vspace{-4pt}
\section{Education}
\vspace{12pt}
University of Texas at Dallas \hfill 2013-2017 \\
\vspace{-12pt}
	\begin{itemize} \itemsep -2pt
	\item BS Computer Science 2017
	\end{itemize}

\vspace{-4pt}
\section{Experience}
\vspace{12pt}
Software Engineering internship at Apcon
\hfill 2016-2017
	\begin{itemize} \itemsep -2pt
	\item Finding and fixing bugs, performing static code analysis
	\item Migrated software to cloud and improved Linux installation
	\item Developed server software using Adobe Flex and C++
	\end{itemize}

Junior Software Engineer at IXIA/Keysight
\hfill 2018-2020
	\begin{itemize} \itemsep -2pt
	\item Developed C/C++ software to inspect packets and extract meta-data using IPFIX
	\item QA tested regression tests using Robot and in-house traffic generator
	\item Set up Docker/Kubernetes environment to run our product virtually
	\end{itemize}

\vspace{-4pt}
\section{Projects}
\vspace{12pt}

%\vspace{6pt}
%Open Source C++ OpenGL/SDL Game and Physics Engine
%\hfill 2015
%	\begin{itemize} \itemsep -2pt
%	\item Personal project using OpenGL 3.3, SDL2, and Lua
%	\item Working graphical renderer and physics simulator
%	\item (\href{https://github.com/justaddbass/ExclusionEngine}{\texttt{github.com/justaddbass/ExclusionEngine}})
%	\end{itemize}

Fencing Tournament Organizer Web Service
\hfill 2018
	\begin{itemize} \itemsep -2pt
		\item Developed in Python and Flask to run fencing tournaments
		\item Stores results in a database to allow for live viewing of results
		\item Used by the Southwest Intercollegiate Fencing Association
		\item (\href{https://github.com/rkblake/FencingTournamentTool}{\texttt{github.com/rkblake/FencingTournamentTool}})
	\end{itemize}

%\vspace{-8pt}
%Flat-Panel Airborne Radio Control
%\hfill 2017
%	\begin{itemize} \itemsep -2pt
%		\item Senior CS Semester project, team of 5
%		\item Worked with Sponsor, Rockwell Collins to develop software for radio control and maintenance
%	\end{itemize}

\vspace{-8pt}
Solar System Simulator
\hfill 2017
	\begin{itemize} \itemsep -2pt
		\item Simulates orbital mechanics using Newtonian physics
		\item Uses OpenGL and SDL2 to display planets and stars
		\item (\href{https://github.com/rkblake/SolarSim/}{\texttt{github.com/rkblake/SolarSim/}})
	\end{itemize}

%\vspace{-8pt}
%Lua C++ Interface
%\hfill 2017
%	\begin{itemize} \itemsep -2pt
%		\item Reads variables from Lua script into C++
%		\item Useful for easy configuration files and hot swapping code
%		\item  (\href{https://github.com/justaddbass/LuaConf}{\texttt{github.com/justaddbass/LuaConf}})
%		\end{itemize}

%\vspace{6pt}
%Python iTunes Library generator
%\hfill 2014
%	\begin{itemize} \itemsep -2pt
%	\item Uses mutagen and ElementTree to read id3 tags and write to xml
%	\item (\href{https://github.com/justaddbass/iTunesLibGen}{\texttt{github.com/justaddbass/iTunesLibGen}})
%	\end{itemize}

\vspace{-8pt}
ECS Maze Solving Toy Car
\hfill 2014
	\begin{itemize} \itemsep -2pt
	\item Freshman Engineering semester project, team of 5
	\item Uses Raspberry Pi, ultrasonic sensors and a Python script to navigate a maze
	\end{itemize}

\vspace{-8pt}
RTS AI Project
\hfill 2010-2012
\begin{itemize} \itemsep -2pt
	\item Worked in 2 person team on Real-Time Strategy playing AI
	\item Created an AI that manages units and resources
	\item Dortmund University of Technology’s Computational Intelligence and Games (CIG) 2011 Starcraft AI
	competitor (\href{http://ls11-www.cs.tu-dortmund.de/rts-competition/starcraft-cig2011}{\texttt{ls11-www.cs.tu-dortmund.de/rts-competition/starcraft-cig2011}})
\end{itemize}

\vspace{-4pt}
\section{Skills}
\vspace{22pt}
	\begin{itemize} \itemsep -2pt
		\item Languages: C/C++, Python, Javascript, SQL, Erlang \ldots
		\item Frameworks and Libraries: Flask, OpenGL, SDL2, Node.js, Vue.js \ldots
		\item Other Skills: git, gdb, Jenkins, Mantis, Linux, AWS, and Machine learning
	\end{itemize}

\vspace{-4pt}
\section{Miscellaneous}
\vspace{22pt}
	\begin{itemize} \itemsep -2pt
	\item Fluent in English, basic understanding of French and Korean
	\item 10 Years competitive Fencing, previous captain of the UTD Fencing Club
	\item Competed in Cyber security competition with UTD's Computer Security Group
	\item Ask me (\href{https://github.com/rkblake}{\texttt{github.com/rkblake}}) about my other projects!
	\end{itemize}

\end{resume}
\end{document}
